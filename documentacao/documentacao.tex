\documentclass[sublist]{fei}
\usepackage[utf8]{inputenc}
%\usepackage[absolute,overlay]{textpos} % posicionar texto em qualquer lugar

\author{André de Souza Mendes}
\title{Documentação - Modelos de veículos simples}
%\subtitulo{subtítulo}
%\cidade{cidade}
%\instituicao{Instituição de Ensino}

%%%%% -- Nao editar
\newglossaryentry{acro}{name={},description={\nopostdesc},sort=a} %Usado para alinhar a lista de abreviaturas
\newglossaryentry{geral}{name={Geral},description={\nopostdesc},sort=a}
\newglossaryentry{grego}{name={Grego},description={\nopostdesc},sort=b}
\newglossaryentry{sub}{name={Subscrito},description={\nopostdesc},sort=c}

\newacronym[longplural=Computational Aided Design,parent=acro]{cad}{CAD}{Computational Aided Design}
\newacronym[longplural=Centro Universitário da FEI,parent=acro]{fei}{FEI}{Centro Universitário da FEI}



%%%% -- Editar abaixo

\newglossaryentry{pontofrentesub}{parent=sub,type=simbolos,name={\ensuremath{{F}}},sort=F,description={Referente ao eixo dianteiro do caminhão-trator}}
\newglossaryentry{pontotratorsub}{parent=sub,type=simbolos,name={\ensuremath{{T}}},sort=T,description={Referente ao centro de massa do caminhão-trator}}
\newglossaryentry{pontotrassub}{parent=sub,type=simbolos,name={\ensuremath{{R}}},sort=R,description={Referente ao eixo traseiro do caminhão-trator}}
\newglossaryentry{pontoarticulacaosub}{parent=sub,type=simbolos,name={\ensuremath{{A}}},sort=A,description={Referente à articulação entre o caminhão-trator e o semirreboque}}
\newglossaryentry{pontosemirreboquesub}{parent=sub,type=simbolos,name={\ensuremath{{S}}},sort=S,description={Referente ao centro de massa da semirreboque}}
\newglossaryentry{pontoeixosemisub}{parent=sub,type=simbolos,name={\ensuremath{{M}}},sort=M,description={Referente ao eixo do semirreboque}}
%\newglossaryentry{pontofrentesub}{parent=sub,type=simbolos,name={\ensuremath{\mathsf{F}}},sort=F,description={Referente ao eixo dianteiro do caminhão-trator}}
%\newglossaryentry{pontotratorsub}{parent=sub,type=simbolos,name={\ensuremath{\mathsf{T}}},sort=T,description={Referente ao centro de massa do caminhão-trator}}
%\newglossaryentry{pontotrassub}{parent=sub,type=simbolos,name={\ensuremath{\mathsf{R}}},sort=R,description={Ponto que localiza o eixo traseiro do caminhão-trator}}
%\newglossaryentry{pontoarticulacaosub}{parent=sub,type=simbolos,name={\ensuremath{\mathsf{A}}},sort=A,description={Referente a articulação entre o caminhão-trator e o semirreboque}}
%\newglossaryentry{pontosemirreboquesub}{parent=sub,type=simbolos,name={\ensuremath{\mathsf{S}}},sort=S,description={Referente ao centro de massa da semirreboque}}
%\newglossaryentry{pontoeixosemisub}{parent=sub,type=simbolos,name={\ensuremath{\mathsf{M}}},sort=M,description={Referente ao eixo do semirreboque}}

%\newglossaryentry{pontofrentegeral}{parent=geral,type=simbolos,name={\ensuremath{{F}}},sort=F,description={Ponto que localiza o eixo dianteiro do caminhão-trator}}
%\newglossaryentry{pontotratorgeral}{parent=geral,type=simbolos,name={\ensuremath{{T}}},sort=T,description={Ponto que localiza o centro de massa do caminhão-trator}}
%\newglossaryentry{pontotrasgeral}{parent=geral,type=simbolos,name={\ensuremath{{R}}},sort=R,description={Ponto que localiza o eixo traseiro do caminhão-trator}}
%\newglossaryentry{pontoarticulacaogeral}{parent=geral,type=simbolos,name={\ensuremath{{A}}},sort=A,description={Ponto que localiza a articulação entre o caminhão-trator e o semirreboque}}
%\newglossaryentry{pontosemirreboquegeral}{parent=geral,type=simbolos,name={\ensuremath{{S}}},sort=S,description={Ponto que localiza o centro de massa da semirreboque}}
%\newglossaryentry{pontoeixosemigeral}{parent=geral,type=simbolos,name={\ensuremath{{M}}},sort=M,description={Ponto que localiza o eixo do semirreboque}}
\newglossaryentry{pontofrentegeral}{parent=geral,type=simbolos,name={\ensuremath{\mathsf{F}}},sort=F,description={Ponto que localiza o eixo dianteiro do caminhão-trator}}
\newglossaryentry{pontotratorgeral}{parent=geral,type=simbolos,name={\ensuremath{\mathsf{T}}},sort=T,description={Ponto que localiza o centro de massa do caminhão-trator}}
\newglossaryentry{pontotrasgeral}{parent=geral,type=simbolos,name={\ensuremath{\mathsf{R}}},sort=R,description={Ponto que localiza o eixo traseiro do caminhão-trator}}
\newglossaryentry{pontoarticulacaogeral}{parent=geral,type=simbolos,name={\ensuremath{\mathsf{A}}},sort=A,description={Ponto que localiza a articulação entre o caminhão-trator e o semirreboque}}
\newglossaryentry{pontosemirreboquegeral}{parent=geral,type=simbolos,name={\ensuremath{\mathsf{S}}},sort=S,description={Ponto que localiza o centro de massa do semirreboque}}
\newglossaryentry{pontoeixosemigeral}{parent=geral,type=simbolos,name={\ensuremath{\mathsf{M}}},sort=M,description={Ponto que localiza o eixo do semirreboque}}


\newglossaryentry{deriva}{parent=grego,type=simbolos,name={\ensuremath{\alpha}},sort=alfa,description={Ângulo de deriva [$rad$]}}

\newglossaryentry{modvel}{parent=geral,type=simbolos,name={\ensuremath{v}},sort=v,description={Módulo do vetor velocidade [$m/s$]}}

\newglossaryentry{vetvel}{parent=geral,type=simbolos,name={\ensuremath{\mathbf{v}}},sort=v,description={Vetor velocidade [$m/s$]}}
\newglossaryentry{vetacel}{parent=geral,type=simbolos,name={\ensuremath{\mathbf{a}}},sort=a,description={Vetor aceleração [$m/s$]}}

\newglossaryentry{vetforca}{parent=geral,type=simbolos,name={\ensuremath{\mathbf{F}}},sort=F,description={Vetor força [$N$]}}
\newglossaryentry{modforca}{parent=geral,type=simbolos,name={\ensuremath{{F}}},sort=F,description={Módulo do vetor força [$N$]}} 


\newglossaryentry{long}{parent=sub,type=simbolos,name={\ensuremath{x}},sort=x,description={Direção longitudinal}}
\newglossaryentry{trans}{parent=sub,type=simbolos,name={\ensuremath{y}},sort=y,description={Direção transversal}}
\newglossaryentry{vert}{parent=sub,type=simbolos,name={\ensuremath{z}},sort=z,description={Direção vertical}}

\newglossaryentry{vetunittrator}{parent=geral,type=simbolos,name={\ensuremath{\mathbf{t}}},sort=t,description={Vetor unitário da base solidária ao caminhão-trator}}
\newglossaryentry{vetunitsemirreboque}{parent=geral,type=simbolos,name={\ensuremath{\mathbf{s}}},sort=s,description={Vetor unitário da base solidária ao semirreboque}}
\newglossaryentry{vetunitfrente}{parent=geral,type=simbolos,name={\ensuremath{\mathbf{e}}},sort=e,description={Vetor unitário da base solidária ao eixo dianteiro}}

%\newglossaryentry{veti}{parent=geral,type=simbolos,name={\ensuremath{\imath}},sort=i,description={Vetor unitário na direção longitudinal}}
%\newglossaryentry{vetj}{parent=geral,type=simbolos,name={\ensuremath{\jmath}},sort=j,description={Vetor unitário na direção transversal}}
%\newglossaryentry{vetk}{parent=geral,type=simbolos,name={\ensuremath{\kmath}},sort=k,description={Vetor unitário na direção vertical}}

\newglossaryentry{massa}{parent=geral,type=simbolos,name={\ensuremath{m}},sort=m,description={Massa [$kg$]}}
\newglossaryentry{momentoinercia}{parent=geral,type=simbolos,name={\ensuremath{I}},sort=i,description={Momento de inércia [$kg \cdot m^2$]}}

\newglossaryentry{dist1}{parent=geral,type=simbolos,name={\ensuremath{a}},sort=a,description={Distância entre o eixo dianteiro e o centro de massa do caminhão-trator [$m$]}}
\newglossaryentry{dist2}{parent=geral,type=simbolos,name={\ensuremath{b}},sort=b,description={Distância entre o centro de massa e o eixo traseiro do caminhão-trator [$m$]}}
\newglossaryentry{dist3}{parent=geral,type=simbolos,name={\ensuremath{c}},sort=c,description={Distância entre o eixo traseiro do caminhão trator e a articulação [$m$]}}
\newglossaryentry{dist4}{parent=geral,type=simbolos,name={\ensuremath{d}},sort=d,description={Distância entre a articulação e o centro de massa do semirreboque [$m$]}}
\newglossaryentry{dist5}{parent=geral,type=simbolos,name={\ensuremath{e}},sort=e,description={Distância entre o centro de massa e o eixo do semirreboque [$m$]}}

%\newglossaryentry{dist1}{parent=geral,type=simbolos,name={\ensuremath{l_{FT}}},sort=AT,description={Distância entre o eixo dianteiro e o centro de massa do caminhão-trator (ponto F e T) [$m$]}}
%\newglossaryentry{dist2}{parent=geral,type=simbolos,name={\ensuremath{l_{TR}}},sort=BT,description={Distância entre o centro de massa e o eixo traseiro do caminhão-trator (ponto T e R), [$m$]}}
%\newglossaryentry{dist3}{parent=geral,type=simbolos,name={\ensuremath{l_{TA}}},sort=E,description={Distância entre o eixo traseiro do caminhão trator e a articulação (ponto R e A) [$m$]}}
%\newglossaryentry{dist4}{parent=geral,type=simbolos,name={\ensuremath{l_{AS}}},sort=AS,description={Distância entre a articulação e o centro de massa do semirreboque (ponto A e S) [$m$]}}
%\newglossaryentry{dist5}{parent=geral,type=simbolos,name={\ensuremath{l_{SM}}},sort=BS,description={Distância entre o centro de massa e o eixo do semirreboque (ponto S e M) [$m$]}}




\newglossaryentry{orientacaotrator}{parent=grego,type=simbolos,name={\ensuremath{\psi}},sort=psi,description={Orientação do caminhão-trator [$rad$]}}
\newglossaryentry{orientacaorelativa}{parent=grego,type=simbolos,name={\ensuremath{\phi}},sort=fi,description={Orientação do semirreboque em relação ao caminhão trator [$rad$]}}
\newglossaryentry{estercamento}{parent=grego,type=simbolos,name={\ensuremath{\delta}},sort=delta,description={Ângulo de esterçamento do eixo dianteiro do caminhão-trator [$rad$]}}
\newglossaryentry{orientacaosemirreboque}{parent=grego,type=simbolos,name={\ensuremath{\gamma}},sort=gama,description={Orientação do semirreboque [$rad$]}}
\newglossaryentry{orientacaovelfrente}{parent=grego,type=simbolos,name={\ensuremath{\beta}},sort=beta,description={Ângulo formado pelo vetor velociade do eixo dianteiro e o eixo longitudinal do módulo dianteiro. [$rad$]}}


\newglossaryentry{coefeqparam}{parent=geral,type=simbolos,name={\ensuremath{q}},sort=q,description={Constante dependente dos parâmetros do veículo}}
\newglossaryentry{expeqestados}{parent=geral,type=simbolos,name={\ensuremath{Q}},sort=Q,description={Expressão algébrica em função dos estados do sistema}}

\newglossaryentry{indice1}{parent=geral,type=simbolos,name={\ensuremath{i}},sort=i,description={Índice}}
\newglossaryentry{indice2}{parent=geral,type=simbolos,name={\ensuremath{j}},sort=j,description={Índice}}

\newglossaryentry{nominal}{parent=sub,type=simbolos,name={\ensuremath{n}},sort=n,description={Nominal}}
\newglossaryentry{homogenea}{parent=sub,type=simbolos,name={\ensuremath{h}},sort=h,description={Homogênea}}


\newglossaryentry{inicial}{parent=sub,type=simbolos,name={\ensuremath{0}},sort=0,description={Inicial}}
\newglossaryentry{equivalente}{parent=sub,type=simbolos,name={\ensuremath{eq}},sort=equivalente,description={Equivalente}}

\newglossaryentry{coefatrito}{parent=grego,type=simbolos,name={\ensuremath{\mu}},sort=mu,description={Coeficiente de atrito}}

\newglossaryentry{coefrigidez}{parent=geral,type=simbolos,name={\ensuremath{C}},sort=C,description={Coeficiente de rigidez lateral [$N/rad$]}}

% O label dos coeficientes a seguir são dados pelo Pacejka2006
\newglossaryentry{by0}{parent=geral,type=simbolos,name={\ensuremath{B}},sort=B,description={Fator de rigidez do modelo de pneu [$N/rad$]}}
\newglossaryentry{dy0}{parent=geral,type=simbolos,name={\ensuremath{D}},sort=D,description={Fator de pico do modelo de pneu [$N$]}}
\newglossaryentry{cy}{parent=geral,type=simbolos,name={\ensuremath{K}},sort=K,description={Coeficiente experimental do modelo de pneu}}
\newglossaryentry{ey}{parent=geral,type=simbolos,name={\ensuremath{E}},sort=E,description={Coeficiente experimental do modelo de pneu}}
\newglossaryentry{c1}{parent=geral,type=simbolos,name={\ensuremath{g}},sort=g,description={Coeficiente experimental do modelo de pneu}}
\newglossaryentry{c2}{parent=geral,type=simbolos,name={\ensuremath{h}},sort=h,description={Coeficiente experimental do modelo de pneu}}


\newglossaryentry{vetestados}{parent=geral,type=simbolos,name={\ensuremath{\mathbf{x}}},sort=x,description={Vetor de estados}}
\newglossaryentry{estado}{parent=geral,type=simbolos,name={\ensuremath{x}},sort=x,description={Estado do sistema}}

\newglossaryentry{funcaovet}{parent=geral,type=simbolos,name={\ensuremath{\mathbf{f}}},sort=f,description={Função vetorial}}
\newglossaryentry{dimensao}{parent=geral,type=simbolos,name={\ensuremath{n}},sort=n,description={Dimensão do sistema}}
\newglossaryentry{vetposorbita}{parent=geral,type=simbolos,name={\ensuremath{\mathbf{r}}},sort=r,description={Vetor posição da órbita vizinha em relação à órbita de referência}}
\newglossaryentry{modposorbita}{parent=geral,type=simbolos,name={\ensuremath{r}},sort=r,description={Módulo do vetor posição da órbita vizinha em relação à órbita de referência}}
\newglossaryentry{tempo}{parent=geral,type=simbolos,name={\ensuremath{t}},sort=t,description={Variável tempo [$s$]}}
\newglossaryentry{base}{parent=geral,type=simbolos,name={\ensuremath{\mathrm{b}}},sort=b,description={Base exponencial}}
\newglossaryentry{lyapunov}{parent=grego,type=simbolos,name={\ensuremath{\lambda}},sort=lambda,description={Expoente de Lyapunov}}
\newglossaryentry{autovalor}{parent=grego,type=simbolos,name={\ensuremath{\Lambda}},sort=Lambda,description={Autovalor da matriz dinâmica}}
\newglossaryentry{autovetor}{parent=geral,type=simbolos,name={\ensuremath{\mathbf{q}}},sort=Lambda,description={Autovetor da matriz dinâmica}}

\newglossaryentry{mapa}{parent=geral,type=simbolos,name={\ensuremath{\mathbf{M}}},sort=M,description={Mapa da solução do sistema de equações diferenciais}}
\newglossaryentry{jacobimapa}{parent=geral,type=simbolos,name={\ensuremath{\mathbf{W}}},sort=W,description={Matriz jabobiana do mapa}}
\newglossaryentry{jacobisist}{parent=geral,type=simbolos,name={\ensuremath{\mathbf{J}}},sort=J,description={Matriz jabobiana do sistema}}
\newglossaryentry{jacobisistelement}{parent=geral,type=simbolos,name={\ensuremath{J}},sort=J,description={Elemento da matriz jabobiana do sistema}}

\newglossaryentry{funcdosist}{parent=geral,type=simbolos,name={\ensuremath{f}},sort=f,description={Função que compõe a função vetorial}}

\newglossaryentry{partereal}{parent=grego,type=simbolos,name={\ensuremath{\sigma}},sort=sigma,description={Parte real do número complexo}}
\newglossaryentry{parteimag}{parent=grego,type=simbolos,name={\ensuremath{\omega}},sort=omega,description={Parte imaginária do número complexo}}



\newglossaryentry{matriza}{parent=geral,type=simbolos,name={\ensuremath{\mathbf{A}}},sort=A,description={Matriz dinâmica do sistema linear}}
\newglossaryentry{matrizb}{parent=geral,type=simbolos,name={\ensuremath{\mathbf{B}}},sort=B,description={Matriz de entradas do sistema linear}}
\newglossaryentry{matrizc}{parent=geral,type=simbolos,name={\ensuremath{\mathbf{C}}},sort=C,description={Matriz de saídas do sistema linear}}
\newglossaryentry{matrizd}{parent=geral,type=simbolos,name={\ensuremath{\mathbf{D}}},sort=D,description={Matriz de transmissão direta do sistema linear}}
\newglossaryentry{vetentradas}{parent=geral,type=simbolos,name={\ensuremath{\mathbf{u}}},sort=u,description={Vetor de entradas}}
\newglossaryentry{vetsaidas}{parent=geral,type=simbolos,name={\ensuremath{\mathbf{y}}},sort=y,description={Vetor de saídas}}


\newglossaryentry{raioinicial}{parent=geral,type=simbolos,name={\ensuremath{\mathbf{p}}},sort=p,description={Vetor de raio inicial da base ortonormal}}
\newglossaryentry{tempoite}{parent=grego,type=simbolos,name={\ensuremath{\tau}},sort=tau,description={Intervalo de tempo em cada iteração [$s$]}}
\newglossaryentry{raioantes}{parent=geral,type=simbolos,name={\ensuremath{\mathbf{w}}},sort=w,description={Vetor que compõe a matriz jacobiana do mapa antes da ortonormalização de Gram-Schmidt}}
\newglossaryentry{raiodepois}{parent=geral,type=simbolos,name={\ensuremath{\bar{\mathbf{w}}}},sort=w,description={Vetor que compõe a matriz jacobiana do mapa depois da ortonormalização de Gram-Schmidt}}

\newglossaryentry{vetdenominadores}{parent=geral,type=simbolos,name={\ensuremath{P}},sort=P,description={Vetor que armazena os denominadores do vetor dos raios a cada iteração}}
\newglossaryentry{numite}{parent=geral,type=simbolos,name={\ensuremath{s}},sort=s,description={Número de iterações do algoritmo para o cálculo dos expoentes de Lyapunov}}
\newglossaryentry{mattransestado}{parent=grego,type=simbolos,name={\ensuremath{\boldsymbol{\Phi}}},sort=Fi,description={Matriz de transição de estado do modelo linear}}
\newglossaryentry{matmodal}{parent=geral,type=simbolos,name={\ensuremath{\mathbf{H}}},sort=H,description={Matriz modal com as colunas compostas pelos autovetores}}
\newglossaryentry{matrespmodais}{parent=geral,type=simbolos,name={\ensuremath{e^{\boldsymbol{\Lambda} t}}},sort=E,description={Matriz diagonal de respostas modais}}


\newglossaryentry{identidade}{parent=geral,type=simbolos,name={\ensuremath{\mathbf{I}}},sort=I,description={Matriz identidade}}

\makeindex
\makeglossaries

\begin{document}

\maketitle

% \begin{folhaderosto}
% Dissertação de Mestrado apresentada ao Centro Universitário da FEI para obtenção do título de Mestre em Engenharia Mecânica, orientado pelo Prof. Dr. Agenor de Toledo Fleury.
% \end{folhaderosto}

% \fichacatalografica % procura o arquivo "ficha.pdf"
% \folhadeaprovacao % procura o arquivo "ata.pdf"
% \dedicatoria{A quem eu quero dedicar o texto.}

% \begin{agradecimentos}
% Obrigado
% \end{agradecimentos}

% \epigrafe{Science is more than a body of knowledge, it’s a way of thinking. A way of skeptically interrogating the universe with a fine understanding of human fallibility.}{Carl Sagan}

%\nocite{dyson_disturbing_1979}

\begin{resumo}
Modelos de veículos simples em Matlab
\palavraschave{Modelo. Veículo. Matlab.}
\end{resumo}

\begin{abstract}
Abstract
\keywords{Keywords. Go. Here.}
\end{abstract}

{
	\pagestyle{empty}
	\listoffigures
	\listoftables
%	\listofalgorithms
%	\glsaddall
%    \printglossaries
    \tableofcontents
}

\chapter{INTRODUÇÃO} 

=================================================

PROXIMAS IMPLEMENTAÇÕES

\begin{itemize}
\item formatar os códigos: - passagem de parametros pros integradores - pastas com cada tipo de analise
\item Implementar a regiao de estabilidade para o pacejka linear
\item implementar a comparacao da regiao de estabilidade
\item implementar o codigo do veiculo totalmente nao linear porem com velocidade prescrita
\end{itemize}

PROXIMAS SIMULAÇÕES

\begin{itemize}
\item linearsadrilyapunovregiao.m com a e b trocados (veículo sobre esterçante)
\item linearpacejkalyapunovregiao.m com a e b certos. Limitar pelo ALPHAT<90graus para que haja uma região
\item linearpacejkalyapunovregiao.m com a e b trocados! Sem 
\end{itemize}


=================================================

Esta documentação visa dar base para os estudos e simulações realizados para avaliar a aplicação dos modelos de veículos em situações dinâmicas extremadas e em estudos de estabilidade através dos expoentes de lyapunov.

Os modelos usados nesta documentação estão listados na tabela \ref{modelos}.

	\begin{table}[ht!] 
	\caption{Modelos bicicleta}
	\label{modelos}
	\begin{center}
	\begin{tabular}{lrr} 
	\hline
	Modelo		       	& Veículo			  	& Pneu	 				   	\\
	\hline
	linearlinear    	& linear		        & linear			       	\\
	linearsadri 		& linear		        & não linear (sadri) 		\\
	linearpacejka		& linear             	& não linear (pacejka)		\\
	naolinearpacejka	& não linear 			& não linear (pacejka)		\\
	\hline
	\end{tabular}
	\caption*{Fonte: Autor}
	\end{center}
	\end{table}

OBS: Todos os modelos possuem o módulo do vetor velocidade do centro de massa constante.

Nos próximos capítulos é descrito o desenvolvimento de cada modelo.

Os parâmetros de veículos e pneus utilizados como base de todas as análises são retirados do artigo de sadri.

\chapter{Pneu}

Modelos de pneus utilizados nas simulações

\section{linear}

Modelo linear 

\section{Sadri}

Mostra a forma geral da curva característica utilizada por sadri.

Script: pneusadri.m

Mostrar as limitações para ângulos grandes. Falta de representatividade.

\section{Pacejka}

Como o modelo de pneu apresentado por sadri apresenta diferenças com relação à representação da realidade, vamos verificar o modelo de pacejka.

Mostra a forma geral da curva característica apresentada por pacejka.

Script: pneupacejka.m

Os efeitos da distribuição de carga nos eixos podem ser levados em consideração no modelo. 

Script: pneupacejkaeixos.m

O ajusto dos parâmetros experimentais de pacejka são ajustados para apresentar a equivalência com relação ao modelo de sadri.

Script: pneusadriXpacejka.m

\chapter{Animação}

Os scripts animacao.m e vetor.m descrevem a movimentação do veículo no plano horizontal com os vetores velocidade (F, T e R) ao longo do tempo.


\chapter{linearlinear} 

PROXIMAS IMPLEMENTAÇÕES

\begin{itemize}
\item integrador para retirar outras variáveis dos resultados
\item utilizar a animação nos resultados
\end{itemize}



Script: linearlinear.m

Basta este script para simular pois o sistema é modelado a partir das matrizes dinâmicas A, B, C e D. A integração é feita dentro do próprio script através do comando lsim.

\chapter{Veiculo linear. Pneu sadri} 

pasta: linearsadri

PROXIMAS IMPLEMENTAÇÕES:

\begin{itemize}
\item Alterar o algoritmo para ele ser genérico e não com a função do sistema ja definida
\item Melhorar o código que chama a animação
\end{itemize}

\section{Equações de movimento}

Desenvolvimento das equações de movimento e matriz jacobiana (simbolico).

Script: linearsadrimovimento.m

\section{Pontos fixos}

Obtenção dos pontos fixos do sistema:

Script: linearsadripontofixo.m

\section{Integração}

O script principal é dado por:

Script: linearsadri.m (Principal)

Nele são definidas as constantes do veículo e pneu. Além disso é feita a integração. Por fim, os resultados são apresentados em gráficos. Com autovalor em função do tempo.

A função de integração se encontra em:

Script: linearsadrifun.m (Função)

Um script extra para o cálculo dos autovalores:

Script: linearsadriautovalor.m (Calculadora de autovalores)

\section{Cálculo dos expoentes}

Um script principal para a chamada do algoritmo

Script: linearsadrilyapunov.m (Principal)

A função que é chamada pelo algoritmo

Script: linearsadrilyapunovext.m (Função extendida com a equação variacional)

Algoritmo de wolf aplicado a um sistema de dois graus de liberdade já com a função de veículo definida dentro do algoritmo.

Script: lyapunov2linearsadri.m

Região de estabilidade:

Script: linearsadrilyapunovregiao.m

Depois de gerado o resultado. O script para carregar as variaveis e plotar os resultados.

Script: linearsadrilyapunovregiaoresultados.m

\chapter{Veículo linear. Pneu pacejka} 

pasta: linearpacejka

PROXIMAS IMPLEMENTAÇÕES

\begin{itemize}
\item Estender o script linearpacejka para plotar a variação da parte real do autovalor
\item 
\end{itemize}


A comparação deste modelo pode ser feita integrando o anterior primeiro. Em seguida deve-se integrar esse aqui sem fechar as figuras criadas.

Script que mostra o desenvolvimento das equações de movimento e a obtenção da matriz jacobiana. 

Script: linearpacejkamovimento.m

O script principal define os parâmetros do veículo e do pneu utilizado e por fim integra o sistema para uma condição inicial e esterçamento constante. Como parte do resultado tem a evolucao do autovalor.

	Script: linearpacejka.m

A função com o modelo é dada por:

	Script: linearpacejkafun.m

Calculador a de autovalor do sistema com pneu pacejka

	Script: linearpacejkaautovalor.m

Cálculo dos expoentes de lyapunov são realisados por meio de um script de definição, modelo extendido com equação variacional e algoritmo de wolf.

Definição das variáveis e chamada para o calculo dos expoentes

	Script: linearpacejkalyapunov.m

Equação extendida para calculo dos expoentes:

	Script: linearpacejkalyapunovext.m

O algoritmo implementado se encontra em:

Script: lyapunov2linearpacejka.m

Região de estabilidade é calculada através de;

	Script: linearpacejkalyapunovregiao.m








\chapter{Veículo não linear. Pneu Pacejka} 

Toolbox de processamento simbólico para o desenvolvimento das equações de movimento.

	Script: naolinearandremovimento.m

Script principal para a integração do sistema

	Script: naolinearandre.m

Função com a dinâmica

	Script: naolinearandrefun.m

Cálculo dos expoentes de Lyapunov.

	Script: naolinearandrelyapunov






%\bibliography{referencias}
%\bibliographystyle{plain}

%\printindex % Índice remissivo


\end{document}

